\documentclass[a4paper]{article}

\usepackage{fullpage} % Package to use full page
\usepackage{parskip} % Package to tweak paragraph skipping
\usepackage{tikz} % Package for drawing
\usepackage{amsmath}
\usepackage{hyperref}
\usepackage{mathtools}
\DeclarePairedDelimiter{\ceil}{\lceil}{\rceil}
\DeclarePairedDelimiter{\floor}{\lfloor}{\rfloor}

\title{NTU CSIE 2016 Fall Algortihm 1st Miterm Solutions}
\author{Chih Yao Chang}
\date{2016/11/01}

\begin{document}

\maketitle

\section{Problem 3}

We showed an $O(n)$-time algorithm for finding the $k$-th largest number in an array of $n$ distinct numbers via an initial division of the input into groups of five numbers. What would the time complexity of the algorithm be if the initial group size is (1) three, (2) seven, and (3)$\ceil*{log_2 n}$? Justify your answers.

\begin{enumerate}
\item group size = $3$
\begin{enumerate}
\item $T(n) = T(\frac{1}{3}n) + max(|X_>|, |X_<|) + O(n) = T(\frac{1}{3}n) + T(\frac{2}{3}n) + O(n)$ (1 points)
\item $T(n) = T(\frac{1}{3}n) + T(\frac{2}{3}n) + O(n) \neq O(n)$ (4 points)
\end{enumerate}
\item group size = $7$
\begin{enumerate}
\item $T(n) = T(\frac{1}{7}n) + max(|X_>|, |X_<|) + O(n) = T(\frac{1}{7}n) + T(\frac{5}{7}n) + O(n)$ (1 points)
\item $T(n) = T(\frac{1}{7}n) + T(\frac{5}{7}n) + O(n) = O(n)$ (4 points)
\end{enumerate}
\item group size = $\ceil*{log_2 n}$
\begin{enumerate}
\item $T(n) = T(\frac{n}{\ceil*{log_2 n}}) + max(|X_>|, |X_<|) + O(n) = T(\frac{n}{\ceil*{log_2 n}}) + T((1-\frac{(\floor{\ceil*{log_2 n}+1)/2}}{2 \times\ceil*{log_2 n}})n) + O(n) \leq T(\frac{n}{\ceil*{log_2 n}}) + T((1-\frac{\ceil*{log_2 n}}{4 \times\ceil*{log_2 n}})n) + O(n) = T(\frac{n}{\ceil*{log_2 n}}) + T(\frac{3}{4}n) + O(n)$ (5 points)
\item $T(\frac{n}{\ceil*{log_2 n}}) + T(\frac{3}{4}n) + O(n) = O(n)$ if $\ceil*{log_2 n} > 4$ (5 points)
\end{enumerate}
\end{enumerate}
Please refer slides {\em algo2016fall05} p.31$\sim$34 for the proof of part(a) and p.23$\sim$30 for the proof of part(b).

\section{Problem 4}

Prove of disprove the recurrence relation
\begin{equation*}
T(n) = \begin{cases} 1, & \mbox{if } n \leq 2 \\ \sqrt{n} \cdot T(\sqrt{n}) + n, & \mbox{if } n\mbox{ otherwise} \end{cases} 
\end{equation*}
implies $T(n) = O(n\text{loglog}n)$.

By definition, we have
\begin{align*}
&
\begin{cases} 
T(n) = \sqrt{n} \cdot T(\sqrt{n}) + n \\ 
T(\sqrt{n}) = \sqrt[4]{n} \cdot T(\sqrt[4]{n}) + \sqrt{n}  \\ 
\cdots \\
T(\sqrt[2^k]{n}) = 1,\text{ where }k = \ceil{\text{loglog}n}
\end{cases}
&\text{(10 points)} \\
\Rightarrow
&
\begin{cases} 
T(\sqrt[2^{k-1}]{n}) = 2 + \sqrt[2^{k-1}]{n} \leq 2\times \sqrt[2^(k-1)]{n} \\ 
T(\sqrt[2^{k-2}]{n}) = \sqrt[2^{k-1}]{n} \cdot T(\sqrt[2^{k-1}]{n}) + \sqrt[2^{k-2}]{n} \leq 3\times \sqrt[2^{k-2}]{n}  \\ 
\cdots \\
T(n) = \sqrt{n} \cdot T(\sqrt{n}) + n \leq (k+1)\times n = O(n\text{loglog}n)
\end{cases}
&\text{(10 points)}
\end{align*}

\section{Problem 5}
Define $h_i = log_{2}n-log_{2}t_i\ s.t.\ 
\text{the time of}\ i^{th}\ \text{operation is}\ h_{i}O(1)$.\\
Define potential function\\
$$\Phi_i =
\begin{cases}
0\quad \text{if}\ i=0\\
\sum\limits_{x=0}^{i} 
	[\log_{2}t_x - \log_{2}i]
	\quad \text{if}\ 1\leq i\leq n
\end{cases}
$$ \\
In other words, $\Phi_i-\Phi_{i-1}=\log_{2}t_i - \log_{2}i\quad \text{for}\ i>0$.\\
Let $\hat{h_i} = h_i +\Phi_i -\Phi_{i-1} $ \ be the amortised cost of $i^{th}$\ operations. The total cost:
$$\sum\limits_{i=1}^{n}h_i=\sum\limits_{i=1}^{n}\hat{h_i}+\Phi_0-\Phi_n$$
By definition,\\
\begin{align*}
&\Phi_0-\Phi_n\\
&=-\sum\limits_{x=0}^n (\log_2{t_x}-\log_2i) \\
&=-\left(\sum\limits_{x=0}^{n}\log_2t_x-\sum\limits_{x=0}^{n}\log_2i\right)\\
&=0\\
\end{align*}
And\\
\begin{align*}
&\sum\limits_{i=1}^{n}\hat{h_i}\\
&=\sum\limits_{i=1}^{n}\left[h_i+\Phi_i-\Phi_{i-1}\right]\\
&=\sum\limits_{i=1}^{n}\log_2n-\log_2i\\
&=\sum\limits_{i=1}^{n}\log_2\frac{n}{i}\\
&=O(n) \quad\text{Please refer slides {\em algo2016fall03} p.53$\sim$55 for elaboration of the last equation}
\end{align*}
By using the potential method, the amount of the time of the $n$ operations is\\
$$\left(\sum\limits_{i=1}^{n}h_i\right)O(1)=O(n)$$
\section{Problem 6}
Let $h(x)=\max\{f(x),g(x)\}$\\
we choose $c_1=1$ and $c_2=1$ to satisfy the inequality:\\
$$c_1h(x)\leq h(x)\leq c_2h(x)$$\\
And note that $f(x)$ and $g(x)$ should be non-negative for $x$ large enough.So,\\
$$c_1h(x)\leq h(x)\leq f(x)+g(x)\leq 2h(x)\leq 2c_2h(x)\quad\text{for x large enough}$$\\
For the inequality above, $f(x)+g(x)=\Theta(\max\{f(x),g(x)\})$ holds.
\end{document}