\documentclass[a4paper]{article}

\usepackage{fullpage} % Package to use full page
\usepackage{parskip} % Package to tweak paragraph skipping
\usepackage{tikz} % Package for drawing
\usepackage{amsmath}
\usepackage{hyperref}
\usepackage{mathtools}
\DeclarePairedDelimiter{\ceil}{\lceil}{\rceil}
\DeclarePairedDelimiter{\floor}{\lfloor}{\rfloor}

\title{NTU CSIE 2016 Fall Algortihm 1st Miterm Solutions}
\author{Chih Yao Chang}
\date{2016/11/01}

\begin{document}

\maketitle

\section{Problem 3}

We showed an $O(n)$-time algorithm for finding the $k$-th largest number in an array of $n$ distinct numbers via an initial division of the input into groups of five numbers. What would the time complexity of the algorithm be if the initial group size is (1) three, (2) seven, and (3)$\ceil*{log_2 n}$? Justify your answers.

\begin{enumerate}
\item group size = $3$
\begin{enumerate}
\item $T(n) = T(\frac{1}{3}n) + max(|X_>|, |X_<|) + O(n) = T(\frac{1}{3}n) + T(\frac{2}{3}n) + O(n)$ (1 points)
\item $T(n) = T(\frac{1}{3}n) + T(\frac{2}{3}n) + O(n) \neq O(n)$ (4 points)
\end{enumerate}
\item group size = $7$
\begin{enumerate}
\item $T(n) = T(\frac{1}{7}n) + max(|X_>|, |X_<|) + O(n) = T(\frac{1}{7}n) + T(\frac{5}{7}n) + O(n)$ (1 points)
\item $T(n) = T(\frac{1}{7}n) + T(\frac{5}{7}n) + O(n) = O(n)$ (4 points)
\end{enumerate}
\item group size = $\ceil*{log_2 n}$
\begin{enumerate}
\item $T(n) = T(\frac{n}{\ceil*{log_2 n}}) + max(|X_>|, |X_<|) + O(n) = T(\frac{n}{\ceil*{log_2 n}}) + T((1-\frac{(\floor{\ceil*{log_2 n}+1)/2}}{2 \times\ceil*{log_2 n}})n) + O(n) \leq T(\frac{n}{\ceil*{log_2 n}}) + T((1-\frac{\ceil*{log_2 n}}{4 \times\ceil*{log_2 n}})n) + O(n) = T(\frac{n}{\ceil*{log_2 n}}) + T(\frac{3}{4}n) + O(n)$ (5 points)
\item $T(\frac{n}{\ceil*{log_2 n}}) + T(\frac{3}{4}n) + O(n) = O(n)$ if $\ceil*{log_2 n} > 4$ (5 points)
\end{enumerate}
\end{enumerate}
Please refer slides {\em algo2016fall05} p.31$\sim$34 for the proof of part(a) and p.23$\sim$30 for the proof of part(b).

\section{Problem 4}

Prove of disprove the recurrence relation
\begin{equation*}
T(n) = \begin{cases} 1, & \mbox{if } n \leq 2 \\ \sqrt{n} \cdot T(\sqrt{n}) + n, & \mbox{if } n\mbox{ otherwise} \end{cases} 
\end{equation*}
implies $T(n) = O(n\text{loglog}n)$.

By definition, we have
\begin{align*}
&
\begin{cases} 
T(n) = \sqrt{n} \cdot T(\sqrt{n}) + n \\ 
T(\sqrt{n}) = \sqrt[4]{n} \cdot T(\sqrt[4]{n}) + \sqrt{n}  \\ 
\cdots \\
T(\sqrt[2^k]{n}) = 1,\text{ where }k = \ceil{\text{loglog}n}
\end{cases}
&\text{(10 points)} \\
\Rightarrow
&
\begin{cases} 
T(\sqrt[2^{k-1}]{n}) = 2 + \sqrt[2^{k-1}]{n} \leq 2\times \sqrt[2^(k-1)]{n} \\ 
T(\sqrt[2^{k-2}]{n}) = \sqrt[2^{k-1}]{n} \cdot T(\sqrt[2^{k-1}]{n}) + \sqrt[2^{k-2}]{n} \leq 3\times \sqrt[2^{k-2}]{n}  \\ 
\cdots \\
T(n) = \sqrt{n} \cdot T(\sqrt{n}) + n \leq (k+1)\times n = O(n\text{loglog}n)
\end{cases}
&\text{(10 points)}
\end{align*}

\end{document}